\documentclass[a4paper,parskip=half,10pt]{scrartcl}
\usepackage{graphicx}
\usepackage{amsmath}
\usepackage{amsfonts}
\usepackage{amssymb}
\usepackage{caption}
\usepackage{subfigure}
\usepackage{float}
\usepackage{hyperref}
\usepackage{siunitx}
\usepackage{isotope}
\usepackage{tikz}
\usepackage{mathrsfs}
\begin{document}
\section{taumain.py}
Die Klasse "dataThread" dient zum Auslesen der Daten aus der tauhost.c Datei. Weiterhin benutzen wir die "animThread"-Klasse, um die Werte, die wir aus "dataThread" bekommen, zu plotten. Zur Datenverarbeitung haben wir einen Eventhandler zwischen "animThread" und "dataThread" benutzt.
Mit den presets übergeben wir der tauhost-Datei, die für die verschiedenen Potentiale spezifischen Werte.
Dann kommen die benutzten Variablen:
\begin{itemize}
\item n: Anzahl der Punkte
\item deltat: Zeitunterschied zwischen zwei Punkten
\item deltatau: Entwicklung der Liste über die fiktive Zeit tau
\item h: Parisi-h
\item parisi: Wird der Parisi-Trick benutzt oder nicht
\item entw: Anzahl der Entwicklungsschritte
\item c: eine momentan nicht benutzte Konstante
\item device: ID des Geräts, das benutzt werden soll
\item rpf: Mit rpf=1 werden in jedem Loop die Daten übergeben
\item intime: Wie viele Loops laufen, bevor in fsum und fhsum geschrieben wird
\item loops: Loops pro Entwicklungsschritt
\item inputf: Möglichkeit der Dateieingabe
\item outputf: In welche Datei wird geschrieben
\item acco: Genauigkeit der geschrieben Daten
\end{itemize}
Das Unterprogramm mit den zuvor definierten Parametern wird ausgeführt in Zeile 154.

\section{tauhost.c}
\begin{itemize}
\item bis Zeile 41: Parameter werden eingelesen
\item Zeilen 47-50: Speicherplatz wird reserviert
\item Zeile 56: Wenn deltatau sich ändert, wird deltatau nicht überschrieben, sondern in deltautmp geschrieben
\item Zeilen 77-79: es werden mit Hilfe der Box-Müller-Transformation zwei gaußverteilte Zufallszahlen zwischen 0 und 1 erzeugt, damit wird omega zum ersten Mal berechnet
\item Zeilen 82-172: Inputverarbeitung, falls Datei als Input genutzt wird
\item Zeilen 174-192: Daten werden initialisiert (Seeds für Zufallszahlengeneration werden erstellt)
\item Zeilen 205-255: Infos über das benutzte Gerät
\item Zeilen 264-319: Speicherbuffer auf dem Gerät wird kreiert
\item Zeilen 324-377: Werte werden das erste Mal in den Buffer geschrieben
\item Zeile 381: Programm wird kreiert
\item Zeile 403: Kernel erstellt
\item Zeilen 417-433: Argumente werden an Kernel übergeben
\item Zeile 481: Beginn des Hauptloops
  \item Zeile 483: Kernel ausführen
  \item Zeile 485: Synchronisierung
  \item Zeile 487: Beginn zweiter Loop (Wertausgabe)
\item Zeile 534: Wert für stable wird ausgelesen und der Kernel meldet, ob die Berechnung stabil verlief oder nicht
\item Zeilen 547f: Wert wird fsum und fhsum hinzugefügt
\item Zeile 563: Wenn die Berechnung nicht stabil verlief, dann wird deltatau aus dem Kernel ausgelesen, in deltatautmp geschrieben und das neu berechnete deltatau wird wieder an den Kernel zurückgegeben
\item Zeile 600f: f-Werte aus dem host-Programm werden zurück in den Kernel geschrieben
\item Zeilen 614-630: Möglichkeit in Output-Datei zu schreiben
\item Zeilen 636-666: Speicherplatz wird freigegeben 

\end{itemize}

\end{document}